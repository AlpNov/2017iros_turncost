%%%%%%%%%%%%%%%%%%%%%%%%%%%%%%%%%%%%%%%%%%%%%%%%%%%%%%%%%%%%%%%%%%%%%%%%%%%%%%%%
%2345678901234567890123456789012345678901234567890123456789012345678901234567890
%        1         2         3         4         5         6         7         8

\documentclass[letterpaper, 10 pt, conference]{ieeeconf}  % Comment this line out if you need a4paper

%\documentclass[a4paper, 10pt, conference]{ieeeconf}      % Use this line for a4 paper

\IEEEoverridecommandlockouts                              % This command is only needed if 
                                                          % you want to use the \thanks command

\overrideIEEEmargins                                      % Needed to meet printer requirements.

% See the \addtolength command later in the file to balance the column lengths
% on the last page of the document

% The following packages can be found on http:\\www.ctan.org
%\usepackage{graphics} % for pdf, bitmapped graphics files
%\usepackage{epsfig} % for postscript graphics files
%\usepackage{mathptmx} % assumes new font selection scheme installed
%\usepackage{times} % assumes new font selection scheme installed
%\usepackage{amsmath} % assumes amsmath package installed
%\usepackage{amssymb}  % assumes amsmath package installed

\title{\LARGE \bf
Subset coverage algorithm for a UAV with turn cost, include physical experiments
}


\author{Albert Author$^{1}$ and Bernard D. Researcher$^{2}$% <-this % stops a space
\thanks{*This work was not supported by any organization}% <-this % stops a space
\thanks{$^{1}$Albert Author is with Faculty of Electrical Engineering, Mathematics and Computer Science,
        University of Twente, 7500 AE Enschede, The Netherlands
        {\tt\small albert.author@papercept.net}}%
\thanks{$^{2}$Bernard D. Researcheris with the Department of Electrical Engineering, Wright State University,
        Dayton, OH 45435, USA
        {\tt\small b.d.researcher@ieee.org}}%
}


\begin{document}



\maketitle
\thispagestyle{empty}
\pagestyle{empty}


%%%%%%%%%%%%%%%%%%%%%%%%%%%%%%%%%%%%%%%%%%%%%%%%%%%%%%%%%%%%%%%%%%%%%%%%%%%%%%%%
\begin{abstract}

Coverage algorithm is a well researched area in path planning. However, most technique focus only on distance cost, but ignore turn cost, which is signficant. Introducing turn cost signficantly increases computation time. Planning a coverage path accounting for turn cost is of interest to UAV flying in obstacle filled environments, or in missions where certain sub-areas are of higher interest than the rest.
	
In this paper, we present an algorithm for subset coverage, or partial coverage with penalty for high interest areas. Our algorithm can work in grid graphs as well as polygonal environments. We then verify the algorithm's output with physical experiment using our hexacopter. We obtain the turn cost through physical experiments, measuring the time and energy cost of making 90 degree turns and 180 degree turns, as well as the decelerating and accelerating period. Our results can be used on our mosquitto-zapping UAV, where our hexacopter carry an electrified net used to eleminate mosquito and collect information on their population.

\end{abstract}


%%%%%%%%%%%%%%%%%%%%%%%%%%%%%%%%%%%%%%%%%%%%%%%%%%%%%%%%%%%%%%%%%%%%%%%%%%%%%%%%
\section{INTRODUCTION}

Coverage path planning is a well researched problem with many applications in robotics.
Cleaning (vacuuming) robot such as the iRobot Roomba rely on these algorithm to ensure that they cover as much area as possible before running out of battery.
Our interest for this area concerns with planning a coverage path for an autonomous unmanned aerial vehicle (UAV) equipped with an electrified net to eliminate and take survey of the mosquito population in the area, see Fig.~\ref{fig:mosuav}.
However, existing methods focus on minimzing distance travelled, with minimal attention to the cost of turning the robot.
For most vehicles, travelling in a straight line is much less costly than travelling in a zig zag path of the same length.

\begin{figure}
	\centering
	\begin{overpic}[width=0.9\columnwidth]{pictures/intro_mosquito_uav.jpg}
	\end{overpic}
	\caption{\label{fig:mosuav}
		Our mosquito net equipped UAV, we wish to find a better coverage algorithm for this UAV.
	}
\end{figure}

In \cite{krupkethesis}, Krupke et al. examined different algorithms and heuristics for planning a coverage path while accounting for turn cost.
Realistic data for the energy used during the turns are needed, however. 
In this paper, we will measure the actual energy used while the UAV changes traveling direction, compared to a straight line path.
Addtionally, with the turn cost obtained, we will use (one or more of the methods listed) to generate a flight plan for an autonomous multicopter to verify their algorithms.



\section{RELATED WORK}

We will use hexacopter/quadcopter/multicopter model from [find a paper] to support our experimental work on turning energy of quadcopter.

Use related work section in Dominik's paper for work in the algorithms.
I'm not sure how to talk about the algorithm part.


\section {TURNING COST OF MULTICOPTER}

\subsection{MODEL}

Model for efficiency of a quad rotor. Some possible sources are \cite{driessens2013}.


\subsection{EXPERIMENTS}

Due to the high variability of a quadrotor's configuration, such as the efficiency curves of the motors used, we will measure the energy used during turning of our hexacopter directly.
To comply with our algorithm's grid base approach, we will measure the energy used by our hexacopter as it travels over three equal-sized grids in flight. See Fig.~\ref{fig:gridturn}.

To measure the energy cost, we use a Raspberry Pi with an ADC logging chip to log the voltage accross and current through the battery of our hexacopter during flight.
This log can then be synchronized with the GPS log recorded by our quadcopter during the flight.

\begin{figure}
	\centering
	\begin{overpic}[width=0.9\columnwidth]{pictures/turncost_grid_turn.pdf}
	\end{overpic}
	\caption{\label{fig:gridturn}
		The three travel path during which we will measure energy
		a) a straight path through three grids
		b) a 90 degree turn
		c) a 180 degree turn
	}
\end{figure}



\section {OUR ALOGRITHM}

\subsection {The algorithm}

Need to explain which algorithm/heuristic we are using.

\subsection {Experimental data}

Find the area sweeped out by the UAV in its path.
Compare our algorithm/path against a boustrophedon path trying to cover the same area, or a square spiral.


\section {CONCLUSION}

Our conclusion


\bibliographystyle{IEEEtran}
\bibliography{cite}

\end{document}
